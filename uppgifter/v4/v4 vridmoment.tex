\documentclass[11pt]{article}
\usepackage{../mtu-uppgifter}

\begin{document}
\raggedright
\setcounter{section}{2}
\section{Vridmoment}

\begin{enumerate}[itemsep=2em]
        \item
              \begin{minipage}[t]{0.5\textwidth}
                      Pelle spänner åt en bult på bildäcket. Han använder en nyckel som är 30 cm lång. \textbf{Beräkna vridmomentet som skapas när Pelle drar i skiftnyckeln med 100 N.}
              \end{minipage}
              \hfill
              \begin{adjustbox}{valign=t}
                      \includesvg[width=15em]{skiftnyckel}
              \end{adjustbox}
        \item
              Pelle och hans lillasyster ska gunga på en 3m lång gungbräda. Pelle väger 80 kg och hans lillasyster väger 40 kg. Lillasystern sitter ute vid brädans ände. \textbf{Hur långt från gungbrädans mittpunkt måste Pelle sitta för att gungbrädan ska vara i balans?}
              \begin{center}
                      \includesvg[width=35em]{gungbräda}
              \end{center}

        \item
              Pella har tråkigt på en lektion och balanserar ett suddigummi på sin linjal som hänger över kanten på bänken. Han märker att allting är i balans när linjalens tyngdpunkt är 10 cm från bänkens kant, och suddigummit är 4 cm från bänkens kant. \textbf{Hur mycket väger linjalen om suddigummit väger 30 g?}
              \begin{center}
                \includesvg[width=35em]{linjal}
              \end{center}

\end{enumerate}

\newpage
\section*{Lösningar \thesection.}
\begin{enumerate}[itemsep=2em]
        \item
              \begin{minipage}[t]{0.5\textwidth}
                      $M = F \cdot l = \SI{100}{\newton} \cdot \SI{0.3}{\metre} = \SI{30}{\newton\metre}$
              \end{minipage}
              \hspace{1em}
              \begin{adjustbox}{valign=t}
                      \includesvg[width=15em]{skiftnyckel}
              \end{adjustbox}
        \item
                            $M_{Pelle} = M_{Lillasyster}$\\
                            $M_{Pelle} = F_{Pelle} \cdot l_{Pelle} = F_{Lillasyster} \cdot l_{Lillasyster}$\\
                            $F_{Pelle} = \SI{80}{\kilogram} \cdot \SI{9.82}{\metre\per\second\squared} = \SI{785.6}{\newton}$\\
                            $F_{Lillasyster} = \SI{40}{\kilogram} \cdot \SI{9.82}{\metre\per\second\squared} = \SI{392.8}{\newton}$\\
                            $l_{Pelle} = \SI{3}{\metre} - l_{Lillasyster}$\\
                            $M_{Pelle} = \SI{785.6}{\newton} \cdot (\SI{3}{\metre} - l_{Lillasyster}) = \SI{392.8}{\newton} \cdot l_{Lillasyster}$\\
                            $l_{Lillasyster} = \SI{1.5}{\metre}$\\
                            $l_{Pelle} = \SI{3}{\metre} - \SI{1.5}{\metre} = \SI{1.5}{\metre}$
                  \begin{adjustbox}{valign=t}
                            \includesvg[width=35em]{gungbräda}
                  \end{adjustbox}
              
\end{enumerate}

\end{document}