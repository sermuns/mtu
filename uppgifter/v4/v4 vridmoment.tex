\documentclass[11pt]{article}
\usepackage{../mtu-uppgifter}

\begin{document}
\raggedright
\setcounter{section}{2}
\section{Vridmoment}

\begin{enumerate}[itemsep=2em]
        \item
              \begin{minipage}[t]{0.5\textwidth}
                      Pelle spänner åt en bult på bildäcket. Han använder en nyckel som är 30 cm lång. \textbf{Beräkna vridmomentet som skapas när Pelle drar i skiftnyckeln med 100 N.}
              \end{minipage}
              \hfill
              \begin{adjustbox}{valign=t}
                      \includesvg[width=17em]{skiftnyckel}
              \end{adjustbox}
        \item
              Pelle och hans lillasyster ska gunga på en 3m lång gungbräda. Pelle väger 80 kg och hans lillasyster väger 40 kg. Lillasystern sitter ute vid brädans ände. \textbf{Hur långt från gungbrädans mittpunkt måste Pelle sitta för att gungbrädan ska vara i jämvikt?}
              \begin{center}
                      \includesvg[width=25em]{gungbräda}
              \end{center}

        \item
              Pelle har tråkigt på en lektion och balanserar ett suddigummi på sin linjal som hänger över kanten på bänken. Han märker att allting är i balans när linjalens tyngdpunkt är 10 cm från bänkens kant, och suddigummit är 4 cm från bänkens kant. \textbf{Hur mycket väger linjalen om suddigummit väger 30 g?}
              \begin{center}
                      \includesvg[width=30em]{linjal}
              \end{center}

        \item
              \begin{minipage}[t]{0.5\textwidth}
                      Pelle har varit snäll och handlat mat åt sin familj. Han hänger matkassarna på styret på sin cykel. Den ena matkassen väger 5 kg och hänger 20 cm från styrets mitt. Den andra matkassen väger 3 kg och hänger 50 cm från styrets mitt. \textbf{Kommer kassarna att hänga i balans, eller kommer Pelle behöva tillföra vridmoment? (Hur mycket, och åt vilken riktning isåfall?)}
              \end{minipage}
              \hfill
              \begin{adjustbox}{valign=t}
                      \includesvg[width=0.4\textwidth]{cykel}
              \end{adjustbox}

        \item
              En 50 ton bro ligger på två stödytor och är i jämvikt.  Avstånd mellan tyngdpunkt och vänster stödyta är 10 meter och avståndet mellan tyngdpunkt och höger stödyta är 5 meter.
              \textbf{Hur mycket kraft tar varje stödyta upp?}
              \begin{center}
                      \includesvg[width=0.9\textwidth]{bro}
              \end{center}

        \item
              Pelle har fått sommarjobb som byggarbetare. På kafferasten sitter han ute på en balk som spänns upp av en lina som är fäst i balkens ände. Linan bildar 30\degree{} mot horisontallinjen. Pelle väger fortfarande 80 kg och balken är 5 meter lång och väger 100 kg. \textbf{Hur mycket kraft tar linan upp?}
              \begin{center}
                      \includesvg[width=0.6\textwidth]{byggarbete}
              \end{center}

\end{enumerate}
\end{document}