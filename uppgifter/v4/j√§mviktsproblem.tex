\documentclass[11pt]{article}

\usepackage{vecka}

\usepackage{../mtu-uppgifter}
\begin{document}

\section*{\lessonNumber. \lessonName}

\begin{enumerate}[itemsep=2em]
        \item
              \begin{minipage}[t]{0.6\textwidth}
                      Pelles vikt är 80 kg, och han lyfter en skivstång som totalt väger 50 kg. \textbf{Rita ut} och \textbf{beräkna} de krafter som verkar precis när Pelle är i toppen på sitt lyft.
              \end{minipage}
              \hspace{2em}
              \begin{adjustbox}{valign=t}
                      \includesvg[width=8em]{pelle}
              \end{adjustbox}
        \item
              \begin{minipage}[t]{0.6\textwidth}
                      En stege som väger 20 kg och ligger mot en vägg. Den har en friktionskraft mot marken på 50 N. \textbf{Rita ut} och \textbf{beräkna} övriga krafter som verkar på stegen.
              \end{minipage}
              \hspace{2em}
              \begin{adjustbox}{valign=t}
                      \includesvg[width=10em]{stege}
              \end{adjustbox}
        \item
              \raggedright
              \setlength{\parskip}{1em}
              Efter att ha gymmat i en timme är Pelle trött och hans kompis måste dra hem honom på en vagn. Snöret som kompisen drar i bildar en vinkel på 30\degree{} mot horisontallinjen. Vagnen väger i princip ingenting. De bromsande krafterna är 25 N.

              \textbf{Hur hårt måste kompisen dra i snöret för att vagnen ska förflyttas med konstant hastighet?}

              \begin{center}
                      \includesvg[width=30em]{vagn}
              \end{center}
\end{enumerate}

\newpage
\section*{Lösningar \thesection.}
\begin{enumerate}[itemsep=2em]
        \item
              \begin{minipage}[t]{0.5\textwidth}
                      Pelles tyngdkraft är $F_{gp} = m_pg$, och skivstångens tyngdkraft är $F_{gs} = m_sg$. Dessa två krafter verkar båda nedåt, men eftersom Pelle och stången fortsätter att stå stilla måste det finnas någon motriktad kraft- detta är normalkraften från marken, $F_{N}$. Först räknar vi ut tyngdkrafterna:
                      \begin{gather*}
                              F_{gp} = m_pg = \SI{80}{\kilogram} \cdot \SI{9.82}{\meter/\second\squared} = \SI{785.6}{\newton} \\
                              F_{gs} = m_sg = \SI{50}{\kilogram} \cdot \SI{9.82}{\meter/\second\squared} = \SI{491}{\newton}
                      \end{gather*}
                      Nu kan vi räkna ut normalkraften, $F_{N}$:
                      \begin{align*}
                              F_{N} - F_{gp} - F_{gs} & = 0                                       \\
                              F_{N}                   & = F_{gp} + F_{gs}                         \\
                              F_{N}                   & = \SI{785.6}{\newton} + \SI{491}{\newton} \\
                              F_{N}                   & = \SI{1276.6}{\newton}
                      \end{align*}
              \end{minipage}
              \hspace{2em}
              \begin{adjustbox}{valign=t}
                      \includesvg[width=15em]{pelle_facit}
              \end{adjustbox}

              \textbf{Svar:} Pelles tyngdkraft är 785.6 N, skivstångens tyngdkraft är 491 N, och normalkraften från marken är 1276.6 N.

              \newpage
        \item
              \begin{minipage}[t]{0.5\textwidth}
                      \raggedright
                      Det finns krafter både i $x$- och $y$-led, så vi gör två ekvationer som ska bli lika med noll. Vi vet att det finns en friktionskraft $F_f$ på 50 N som verkar åt vänster, negativt i $x$-led. Stegens tyngdkraft $F_{g}$ verkar nedåt, negativt i $y$-led.

                      Eftersom stegen rör sig i konstant hastighet (står stilla) i är summan av krafterna i $x$-led lika med noll, och därför finns det någon motriktad kraft till friktionskraften, nämligen normalkraften som från väggen, $F_{nv}$. Summan av krafterna i $y$-led ska också vara lika med noll, och det finns också en normalkraft från marken $F_{nm}$ som motverkar tyngdkraften. Vi får alltså två ekvationer:

              \end{minipage}
              \hfill
              \begin{adjustbox}{valign=t}
                      \includesvg[width=0.4\textwidth]{stege_facit}
              \end{adjustbox}
              \begin{align*}
                      \textbf{x-led}: \begin{aligned}[t]
                                              -F_f + F_{nv} & = 0              \\
                                              -50 + F_{nv}  & = 0              \\
                                              F_{nv}        & = 50\,\mathrm{N}
                                      \end{aligned}
              \end{align*}
              \begin{align*}
                      \textbf{y-led}: \begin{aligned}[t]
                                              -F_g + F_{nm}                                                        & = 0                 \\
                                              -\SI{20}{\kilogram} \cdot \SI{9.82}{\meter/\second\squared} + F_{nm} & = 0                 \\
                                              F_{nm}                                                               & = 196.4\,\mathrm{N}
                                      \end{aligned}
              \end{align*}
              \textbf{Svar:} Normalkraften från väggen är 50 N, \\ och normalkraften från marken är 196.4 N.

              \newpage
        \item
              \raggedright
              Vi börjar med att rita ut alla krafter som verkar på vagnen. Vi vet att vagnen rör sig med konstant hastighet, så summan av krafterna i $x$-led är noll. Det finns en motriktad kraft till de bromsande krafterna, och det är dragkraften $F_d$ som kompisen drar med. Men mer specifikt är det bara x-komponenten av $F_d$ som är verksam. Vinkeln mellan $F_d$ och $x$-axeln är 30\degree{}, alltså blir $F_{dx}$ = $F_d \cdot cos(30\degree)$. Vi vet också att summan av krafterna i $y$-led är noll, eftersom vagnen inte rör sig i höjdled. Det finns en normalkraft $F_{N}$ som motverkar tyngdkraften $F_g$.
              \begin{align*}
                      -F_b + F_{dx}                    & = 0                                      \\
                      -F_b + F_d \cdot \cos(30\degree) & = 0                                      \\
                      F_d \cdot \cos(30\degree)        & = F_b                                    \\
                      F_d                              & = \frac{F_b}{\cos(30\degree)}            \\
                      F_d                              & = \frac{25}{\cos(30\degree)}\,\mathrm{N} \\
                      F_d                              & = 28.87\,\mathrm{N}
              \end{align*}
              \textbf{Svar:} Kompisen måste dra med en kraft på 28.87 N.
\end{enumerate}

\end{document}