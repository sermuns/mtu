\documentclass[11pt]{article}

\usepackage{vecka}
\renewcommand{\facit}{Facit }
\usepackage{../mtu-uppgifter}

\begin{document}
\raggedright

\section*{Facit till \lessonNumber. \lessonName}
\begin{enumerate}[itemsep=2em]
        \item
              \raggedright
              Formeln för att räkna ut vridmoment är $M = F \cdot l$. Vi vet att vridmomentet är \SI{10 000}{\newton\meter} och att kofoten är \SI{65}{\centi\meter} lång. Vi kan lösa ut kraften $F$ ur formeln:
              \begin{equation*}
                      F = \frac{M}{l} = \frac{\SI{10 000}{\newton\meter}}{\SI{65}{\centi\meter}} = \frac{{\SI{10 000}{\newton\meter}}}{{\SI{0.65}{\meter}}} = \SI{15384.6}{\newton} \approx \SI{15.4}{\kilo\newton}
              \end{equation*}
              \textbf{Svar:} Pelle måste använda en kraft på \SI{15.4}{\kilo\newton} för att bryta upp dörren.
        \item
              \raggedright
              Vi vet att soffans massa är \SI{50}{\kilo\gram}. Den kommer att lyftas i konstant hastighet, och är alltså i jämvikt. Vi kan använda formeln $F_g = m \cdot g$ för att räkna soffans tyngdkraft.
              \begin{equation*}
                      F_g = m \cdot g = \SI{50}{\kilo\gram} \cdot \SI{9.82}{\meter\per\second\squared} = \SI{491}{\newton}
              \end{equation*}
              \begin{minipage}[c]{0.4\textwidth}
                      Jämviktsekvation i vertikalled ger:
              \end{minipage}
              \begin{minipage}[c]{0.4\textwidth}
                      \begin{align*}
                              \uparrow : \begin{aligned}[t]
                                                 F_{\text{rep}} - F_g & = 0                 \\
                                                 F_{\text{rep}}       & = F_g               \\
                                                 F_{\text{rep}}       & = \SI{491}{\newton}
                                         \end{aligned}
                      \end{align*}
              \end{minipage}

              \textbf{Svar:} Pelle måste dra i repet med en kraft på \SI{491}{\newton} för att lyfta soffan i konstant hastighet.

        \item

\end{enumerate}
\vfill
\begin{center}
        \fbox{\parbox{0.5\textwidth}{\centering
                        \textbf{Senast ändrad:} \today{}, kl \currenttime
                }}
\end{center}

\end{document}