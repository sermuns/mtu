\documentclass[11pt]{article}

\usepackage{vecka}
\renewcommand{\facit}{Facit }
\usepackage{../mtu-uppgifter}

\begin{document}
\raggedright

\section*{Facit till \lessonNumber. \lessonName}
\begin{enumerate}[itemsep=2em]
        \item
              \raggedright
              Formeln för att räkna ut vridmoment är $M = F \cdot l$. Vi vet att vridmomentet är \SI{10 000}{\newton\meter} och att kofoten är \SI{65}{\centi\meter} lång. Vi kan lösa ut kraften $F$ ur formeln:
              \begin{equation*}
                      F = \frac{M}{l} = \frac{\SI{10 000}{\newton\meter}}{\SI{65}{\centi\meter}} = \frac{{\SI{10 000}{\newton\meter}}}{{\SI{0.65}{\meter}}} = \SI{15384.6}{\newton} \approx \SI{15.4}{\kilo\newton}
              \end{equation*}
              \textbf{Svar:} Pelle måste använda en kraft på \SI{15.4}{\kilo\newton} för att bryta upp dörren.
        \item
              \raggedright
              Vi vet att soffans massa är \SI{50}{\kilo\gram}. Den kommer att lyftas i konstant hastighet, och är alltså i jämvikt. Vi kan använda formeln $F_g = m \cdot g$ för att räkna soffans tyngdkraft.
              \begin{equation*}
                      F_g = m \cdot g = \SI{50}{\kilo\gram} \cdot \SI{9.82}{\meter\per\second\squared} = \SI{491}{\newton}
              \end{equation*}
              \begin{minipage}[c]{0.4\textwidth}
                      Jämviktsekvation i vertikalled ger:
              \end{minipage}
              \begin{minipage}[c]{0.4\textwidth}
                      \begin{align*}
                              \uparrow : \begin{aligned}[t]
                                                 F_{\text{rep}} - F_g & = 0                 \\
                                                 F_{\text{rep}}       & = F_g               \\
                                                 F_{\text{rep}}       & = \SI{491}{\newton}
                                         \end{aligned}
                      \end{align*}
              \end{minipage}

              \textbf{Svar:} Pelle måste dra i repet med en kraft på \SI{491}{\newton} för att lyfta soffan i konstant hastighet.

        \item
              Soffan väger fortfarande \SI{50}{\kilo\gram}, men nu måste Pelle inte motverka hela soffans tyngd, utan endast den del som är parallell med rampen.
              Vi vet att normalkraften från rampen på soffan är \SI{425.22}{\newton}, och från förra uppgiften hittade vi soffans tyngdkraft, \SI{491}{\newton}. Eftersom alla krafter tar ut varandra i jämvikt, så är den kraft som Pelle måste använda för att dra soffan uppför rampen lika stor som den kraft som behövs för att motverka soffans tyngd parallellt med rampen.

              \begin{minipage}[c]{0.7\textwidth}
                      \raggedright
                      Vi kan använda pythagoras sats för att räkna ut repkraften:
              \end{minipage}
              \hfill
              \begin{minipage}[c]{0.15\textwidth}
                      \includesvg[width=0.95\textwidth]{soffa_ramp_triangel}
              \end{minipage}

              \begin{minipage}[c]{0.2\textwidth}
                      \begin{align*}
                              F_{\text{rep}}^2 + F_{\text{n}}^2           & = F_g^2                                                   \\
                              F_{\text{rep}}^2 + (\SI{425.22}{\newton})^2 & = (\SI{491}{\newton})^2                                   \\
                              F_{\text{rep}}^2                            & = (\SI{491}{\newton})^2 - (\SI{425.22}{\newton})^2        \\
                              F_{\text{rep}}                              & = \sqrt{(\SI{491}{\newton})^2 - (\SI{425.22}{\newton})^2} \\
                              F_{\text{rep}}                              & \approx (\SI{245.5}{\newton})
                      \end{align*}
              \end{minipage}
              \hspace{1em}
              \begin{minipage}[c]{0.3\textwidth}
                      \includesvg[width=0.95\textwidth]{soffa_ramp_facit}
              \end{minipage}

              \textbf{Svar:} Pelle måste dra i repet med en kraft på \SI{245.5}{\newton}.
              \newpage
        \item
              \textbf{Svar: } a, b och e är exempel på mekanikens gyllene regel (det man vinner i kraft förloras i väg).

        \item
              Eftersom Pelle cyklar i konstant hastighet (30 km/h) är hela systemet i jämvikt. Vi kan sätta upp en momentekvation med bakhjulet $B$ som momentpunkt.

              \begin{align*}
                      \overset{\curvearrowleft}{B} : \begin{aligned}[t]
                                                             -F_{\text{pelle}} \cdot \SI{1.2}{\meter} + F_{\text{cykel}} \cdot \SI{1.2}{\meter} + F_{\text{fram}} \cdot (\SI{1.2}{\meter} + \SI{0.3}{\meter})                                                                  & = 0                                                   \\
                                                             -\SI{80}{\kilo\gram} \cdot \SI{9.82}{\meter/\second\squared} \cdot \SI{1.2}{\meter} - \SI{10}{\kilo\gram} \cdot \SI{9.82}{\meter/\second\squared} \cdot \SI{1.2}{\meter} + F_{\text{fram}} \cdot \SI{1.5}{\meter} & = 0                                                   \\
                                                             \SI{-942.72}{\newton\meter} - \SI{117.84}{\newton\meter} + F_{\text{fram}} \cdot \SI{1.5}{\meter}                                                                                                                 & = 0                                                   \\
                                                             F_{\text{fram}} \cdot \SI{1.5}{\meter}                                                                                                                                                                            & = \SI{1060.6}{\newton\meter}                          \\
                                                             F_{\text{fram}}                                                                                                                                                                                                   & = \frac{\SI{1060.6}{\newton\meter}}{\SI{1.5}{\meter}} \\
                                                             F_{\text{fram}}                                                                                                                                                                                                   & = \SI{707.07}{\newton}
                                                     \end{aligned}
              \end{align*}
              Vi kan nu använda jämviktsekvationen i vertikalled för att räkna ut kraften i bakhjulet:
              \begin{align*}
                      \uparrow : \begin{aligned}[t]
                                         - F_{\text{cykel}} - F_{\text{pelle}} + F_{\text{bak}} + F_{\text{fram}} & = 0                                                                                                                                                \\
                                         F_{\text{bak}}                                                           & = F_{\text{cykel}} + F_{\text{pelle}} - F_{\text{fram}}                                                                                            \\
                                         F_{\text{bak}}                                                           & = \SI{80}{\kilo\gram} \cdot \SI{9.82}{\meter/\second\squared} + \SI{10}{\kilo\gram} \cdot \SI{9.82}{\meter/\second\squared} - \SI{707.07}{\newton} \\
                                         F_{\text{bak}}                                                           & = \SI{176.73}{\newton}
                                 \end{aligned}
              \end{align*}

              \textbf{Svar: } Bakdäcket tar upp en kraft på \SI{176.73}{\newton} och framdäcket tar upp en kraft på \SI{707.07}{\newton}.

              \newpage
        \item
              Vi kan sätta upp en momentekvation kring kanten $K$ för att räkna ut kraften $F$ som Pelle måste använda för att välta tillbaka bilen på hjulen igen.
              \begin{align*}
                      \overset{\curvearrowleft}{K} : \begin{aligned}[t]
                                                             -F \cdot \SI{2.9}{\meter} + F_{\text{bil}} \cdot \SI{0.1}{\meter}                                                 & = 0                                                  \\
                                                             -F \cdot \SI{2.9}{\meter} + \SI{2 000}{\kilo\gram} \cdot \SI{9.82}{\meter/\second\squared} \cdot \SI{0.1}{\meter} & = 0                                                  \\
                                                             -F \cdot \SI{2.9}{\meter} + \SI{1 964}{\newton\meter}                                                             & = 0                                                  \\
                                                             -F \cdot \SI{2.9}{\meter}                                                                                         & = -\SI{1 964}{\newton\meter}                         \\
                                                             F                                                                                                                 & = \frac{\SI{1 964}{\newton\meter}}{\SI{2.9}{\meter}} \\
                                                             F                                                                                                                 & \approx \SI{677.24}{\newton}
                                                     \end{aligned}
              \end{align*}
              \begin{center}
                      \includesvg[width=0.7\textwidth]{hävarm_bil_facit}
              \end{center}




\end{enumerate}
\vfill
\begin{center}
        \fbox{\parbox{0.5\textwidth}{\centering
                        \textbf{Senast ändrad:} \today{}, kl \currenttime
                }}
\end{center}

\end{document}