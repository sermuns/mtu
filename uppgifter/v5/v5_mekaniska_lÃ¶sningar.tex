\documentclass[11pt]{article}

\usepackage{vecka}

\usepackage{../mtu-uppgifter}

\begin{document}
\raggedright

\section*{\lessonNumber. \lessonName}
\begin{enumerate}[itemsep=2em]
        \item
              \begin{minipage}[t]{0.5\textwidth}
                      Pelle glömde att lämna in sin skoluppgift.
                      Under natten bryter han sig in i skolan för att smyga in uppgiften i lärarens pärm. Det krävs \SI{10 000}{\newton\meter} för att bryta upp dörren till skolan. Pelle har en kofot som är \SI{65}{\centi\meter} lång. \textbf{Hur stor kraft måste Pelle använda för att bryta upp dörren?}
              \end{minipage}
              \hspace{1em}
              \begin{adjustbox}{valign=t}
                      \includesvg[width=0.4\textwidth]{kofot}
              \end{adjustbox}
        \item
              \begin{minipage}[t]{0.5\textwidth}
                      Pelle ska hjälpa sin kompis flytta upp en soffa. Den väger \SI{50}{\kilo\gram}. Han väljer att lyfta soffan rakt uppåt med ett rep. \textbf{Med hur mycket kraft måste Pelle dra i repet för att lyfta soffan i konstant hastighet?}
              \end{minipage}
              \hspace{1em}
              \begin{adjustbox}{valign=t}
                      \includesvg[width=7em]{soffa_upp}
              \end{adjustbox}

        \item
              Det var för tungt för Pelle att lyfta soffan hela vägen upp. Hans kompis kom på idén att bygga en ramp. Normalkraften från rampen på soffan är \SI{425.22}{\newton} \textbf{Hur stor kraft måste Pelle nu använda för att dra soffan uppför rampen i konstant hastighet?}
              \vspace{0.5em}
              \begin{center}
                      \includesvg[width=20em]{soffa_ramp}
              \end{center}
              \newpage

        \item
              I vilka av följande situationer visar sig mekanikens gyllene regel?
              \begin{enumerate}[label=\textbf{\alph*.}]
                      \item En lång nyckel gör det enklare att dra åt en bult jämfört med en kort nyckel.
                      \item Det är jobbigare i stunden att cykla uppför en brant backe jämfört med en planare backe.
                      \item Verktyg som drivs av elektricitet är starkare än verktyg som drivs av handkraft.
                      \item Batterier som används sällan håller längre än batterier som används ofta.
                      \item Pincett gör det enklare att nypa hårdare än med fingrarna.
              \end{enumerate}

        \item
              \begin{minipage}[t]{0.5\textwidth}
                      Pelle är ute och cyklar 30 km/h. Han väger 80 kg, och sitter precis på cykelns tyngdpunkt. Cykeln väger 10 kg. \textbf{Hur mycket kraft tar bak- respektive framhjulet upp?}
              \end{minipage}
              \hspace{1em}
              \begin{adjustbox}{valign=t}
                      \includesvg[width=0.35\textwidth]{fackverk_cykel}
              \end{adjustbox}

        \item
              Pelle åkte och välte med farsans bil. Han försöker välta tillbaka bilen på hjulen igen. Bilen väger 2 ton och till sin hjälp har han en stålbalk som är 3 meter lång, varav 10 cm kan han få in under bilen. \textbf{Hur stor kraft måste Pelle använda för att välta tillbaka bilen på hjulen igen?}
              \begin{center}
                      \includesvg[width=0.8\textwidth]{hävarm_bil}
              \end{center}


\end{enumerate}
\end{document}